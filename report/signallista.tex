\section{Signallista}
\subsection{DS18S20}
\begin{table}[H]
	\begin{tabularx}{\textwidth}{p{2cm} p{4.3cm} X}
		\hline
		\textbf{Namn} & \textbf{Typ} & \textbf{Kommentar} \\
		\hline
		clk & in std\_ulogic & Global klocka, 100MHz \\
		reset & in std\_ulogic & Global asynkron reset \\
		measure & in std\_ulogic & Påbörja temperaturavläsning \\
		valid & buffer std\_ulogic & Temperaturavläsning klar. 
			Giltig så länge \signal{valid = '1'} \\
		DQ & inout std\_logic & Pinne till sensor \\
		temperature & buffer signed(7 downto 0) & Temperatur i tvåkompliments binärform \\
		\hline
	\end{tabularx}
\end{table}


\subsection{segment\_temperature}
\begin{table}[H]
	\begin{tabularx}{\textwidth}{p{2cm} p{4.3cm} X}
		\hline
		\textbf{Namn} & \textbf{Typ} & \textbf{Kommentar} \\
		\hline
		clk & in std\_ulogic & Global klocka, 100MHz \\
		reset & in std\_ulogic & Global asynkron reset \\
		rawd & in signed(7 downto 0) & Temperatur i tvåkompliments binärform \\
		valid & out std\_ulogic & Temperaturavläsning klar. 
			Giltig så länge \signal{valid = '1'} \\
		an & buffer std\_ulogic\_vector(3 downto 0) & 7 Segment-enable \\
		segment & buffer std\_ulogic\_vector(7 downto 0) & Utsignal till 7 segmentsbussen \\
		\hline
	\end{tabularx}
\end{table}

\subsection{Com}
\begin{table}[H]
	\begin{tabularx}{\textwidth}{p{2cm} p{4.3cm} X}
		\hline
		\textbf{Namn} & \textbf{Typ} & \textbf{Kommentar} \\
		\hline
		clk & in std\_ulogic & Global klocka, 100MHz \\
		rst & in std\_ulogic & Global asynkron reset \\
		tempInAvail & in std\_logic & Temperatur finns tillgänlig på bus\\
		elementInAvail & in std\_logic & Elementdata finns tillgänlig på bus\\
		requestTemp & out std\_logic & Ber styrenheten om tempdata till bus\\
		requestElement & out std\_logic & Ber styrenheten om elementdata till bus\\
		elementOutAvail & out std\_logic & Meddelar styrenheten, elementdata på bus\\
		tempIn & in std\_logic\_vector(7 downTo 0) & Databus från temperaturmodul\\
		elementIn & in std\_logic\_vector(1 downTo 0) & Databus från elementmodul \\
		elementOut & out std\_logic\_vector(7 downTo 0) & Databus till elementmodul\\
		tx & out std\_logic & Seriel port ut till GSM enhet ut från AT-modul\\
		rx & in std\_logic & Seriel port in till AT-modul från GSM-modul\\		
		\hline
	\end{tabularx}
\end{table}
\subsection{Styrenhet}
\begin{table}[H]
	\begin{tabularx}{\textwidth}{l l X}
		\hline
		\textbf{Namn} & \textbf{Typ} & \textbf{Kommentar} \\
		\hline
		clk & in std\_logic & Global klocka, 100MHz \\
		reset & in std\_logic & Global asynkron reset \\
		tempPutTempOnDb & out std\_logic & Tala om för Temperatur-blocket att lägga ut temperaturen på databussen\\
		tempNowOnDb & in std\_logic & Temperaturen finns tillgänlig på databussen\\
		comHasTempOnDb & out std\_logic & Talar om för Kommunikations-blocket att temperaturen finns på databussen\\
		comHasElemStatusOnDb & out std\_logic & Talar om för Kommunikations-blocket att aktuell elementstatus finns på databussen \\
		comHasElemStatus & in std\_logic & Meddelar styrenheten att Element-blocket har ny elementstatus på bussen\\
		comWantElemStatus & in std\_logic & Talar om för styrenheten att aktuell elementstatus önskas på databussen\\
		comWantTemp & in std\_logic & Talar om för styrenheten att temperaturen önskas på databussen \\
		elemHasStatusOnDb & out std\_logic & Talar om att ny elementstatus finns på databussen\\
		elemPutStatusOnDb & out std\_logic & Talar om att aktuell elementstatus ska skrivas till databussen\\
		elemNewStatusDone & in std\_logic & Talar om att elementstatus har uppdaterats till det som fanns på databussen\\
		elemNewStatusOnDb & in std\_logic & Talar om att aktuell elementstatus finns tillgängligg på databussen\\
		\hline
	\end{tabularx}
\end{table}
\subsection{Element}
\begin{table}[H]
	\begin{tabularx}{\textwidth}{l l X}
		\hline
		\textbf{Namn} & \textbf{Typ} & \textbf{Kommentar} \\
		\hline
		clk & in std\_logic & Global klocka, 100MHz \\
		reset & in std\_logic & Global asynkron reset \\
		setNewStatus & in std\_logic & Reglera elementen enligt ny elementstatus från databussen\\
		returnStatus & in std\_logic & Skriv aktuell elementstatus till databussen\\
		statusOnDb & out std\_logic & Talar om att aktuell elementstatus skrivits till databussen\\
		updatedStatus & out std\_logic & Talar om att elementstatus nu ändrat till det som fanns på databussen \\
		input & in std\_logic\_vector(3:0) & Databuss för inkommand elementstatus\\
		output & out std\_logic\_vector(3:0) & Databuss för utgående aktuell elementstatus\\
		\hline
	\end{tabularx}
\end{table}
